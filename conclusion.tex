\chapter{Conclusion}

It is established that spatial navigation employs two mechanisms for successful encoding of the environment and tracking the location of self: one is based on the information about stable landmarks, cues and environmental boundaries, another enables to compute movement trajectory using orientation and distance traveled coming mainly from self-motion (\cite{OKeefe1978}; \cite{Buzsaki2013}). Both types of information pathways closely interact not only to form a coherent representation of space, but also to maintain a unique identity of the current location, keeping dynamic balance depending on which type of information is better available. In the brain, the hippocampal-entorhinal network is a key structure involved in implementation of the main mechanisms of spatial navigation. While entorhinal cortices provide pre-processed information about self-motion (grid cells, mEC), landmarks (landmark-vector cells, lEC) and boundaries (boundary cells, mEC), hippocampal place cells are known to be able to sequentially integrate the combination of these types of information to form stable episodes, building invariant representation of a certain location in a particular environment (\cite{Moser2008}; \cite{Moser2015}). The mechanisms of this integration are of deep scientific interest since the discovery of place cells (\cite{OKEEFE1971171}), however the exact details of this interaction on both functional and cellular levels are still not known.

I recorded hippocampal activity from randomly foraging gerbils in the freely-moving 3D virtual reality system to  further research this interaction. As virtual reality systems enable to manipulate visual information, inducing a conflict between vision and other sensory systems, recording hippocampal single cell activity in the conflicting conditions allows to explore the mechanisms of place cell formation when their afferent inputs do not match.

I designed experiments that introduce a periodic (shift of the virtual space) or instant (gain between vision and locomotion) conflict between external sensory and internal self-motion systems. The recorded data presents a novel way to show several features of hippocampal neurons, consistent with previous findings:

\begin{itemize}
    \item place cells form their location representation based on visual and self-motion / boundary-defined inputs;
    \item the density of the visually-driven place fields was higher in the middle of the arena, confirming that one reference frame (e.g. visual) dominates when the other is weaker or not instantly available (e.g. arena boundaries);
    \item the overall place cell activity is reduced in darkness;
    \item cells driven mainly by self-motion inputs keep their place selectivity in darkness, while visually-driven cells mostly stop firing or remap;
    \item hippocampal cells might use a weighted combination of sensory inputs while establishing a spatial representation.
\end{itemize}

In line with some previous virtual reality studies (\cite{Chen2013}; \cite{Haas2019}) recorded neurons could be classified in distinct groups depending on the strength of their visual or self-motion afferents. However, I was able to better characterize the group of hippocampal neurons driven by a combination of visual and self-motion signals in freely-moving animals naturally foraging in a 2D environment. By comparing single cell activity in light and dark conditions this multisensory group could be discretized. In both periodic (shift) and instant (gain) conflicting conditions these neurons demonstrate an ability to encode an intermediate position between estimates, defined by two spatial reference frames. This resulted in a few suggestions of the potential mechanisms how this two types of information can be integrated at the level of the CA1 place cells, some of which are consistent with the dynamic loop models of the hippocampal-entorhinal network (\cite{Li2020}).

Hippocampus and entorhinal cortex are not only involved in navigation, but also support formation and maintenance of general declarative memories (\cite{Scoville2000}). The parallel between explicit allocentric representations and items of semantic memory, or between trajectories in space and sequences of episodic events (\cite{Buzsaki2013}) suggests that similar networks and mechanisms are involved in spatial navigation and all other types of sequential actions, like walking, speaking, or mental planning. The discovery of “schema” cells in monkeys is recent evidence that the hippocampus can handle these types of abstract representations (\cite{Baraduc2019}). Mechanisms proposed in the current manuscript could be expanded to the general memory formation also in non-spatial domains. The further research on spatial navigation or on the hippocampus in general is a great potential for the neuroscience field to ultimately understand the key brain mechanisms, particularly such an essential aspect of it as learning and memory.
